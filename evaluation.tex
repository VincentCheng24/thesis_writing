\subsection{Evaluation Metrics}
\label{eval_met}

We use intersection over union (IoU) to measure the performance of 3D vehicle detection. IoU used to measure the similarity of two 2D bounding boxes in various 2D object detection challenges, \eg Pascal VOC \cite{Everingham15} and ILSVRC \cite{DBLP:Russakovsky14}. Recently it is extended to measure 3D object detection with the formula:
\begin{equation}
	IoU(b_1, b_2) = \frac{V(b_{pre}\cap b_{gt})}{V(b_{pre}\cup b_{gt})}
\end{equation}
where $V(\cdot)$ indicates the volume and $b_{pre}$ denotes the predicted 3D bounding box while $b_{gt}$ is the ground truth. KITTI 3D object detection benchmark considers that a 3D object detection is correct, if $IoU \geq 0.7$ \cite{Geiger2012CVPR}. If multiple bounding boxes are predicted for one vehicle, they are considered as false predictions. %{But in our experiment, we lower the criterion to $IoU \geq 0.5$.}% 

For 3D orientation estimation, we use the measure,Orientation Score (OS), defined in \cite{DBLP:journals/corr/MousavianAFK16}. It is the mean error across all estimations in the validation set, written as:
\begin{equation}
	OS =\frac{1}{N} \sum_{i=1}^N\frac{(1+\cos(\triangle \theta_i))}{2}
\end{equation}
where $N$ denotes the number of examples in the validation set and $\triangle \theta_i$ represents the difference between the predicted orientation $r_y$ and the ground truth for example $i$.

 For other tasks, We follow the metrics set by \cite{DBLP:journals/corr/ChabotCRTC17}. A 3D localization is considered correct if its distance to the ground truth is less than a threshold. Two thresholds, 1 meter and 2 meters, are chosen. 2D part localization are measured the same way and the threshold is 20 pixels. 3D dimension estimation is correct if the predicted dimensions $(h, w, l)$ satisfies the equation
\begin{equation}
	\left | \frac{h-h_{gt}}{h_{gt}} \right | < 0.2  ~~\&~~\left | \frac{w-w_{gt}}{w_{gt}} \right | < 0.2  ~~\&~~ \left | \frac{l-l_{gt}}{l_{gt}} \right | < 0.2
\end{equation}
where $(h_{gt}, w_{gt}, l_{gt})$ is the ground truth. 2D part visibility prediction is a pure classification problem so that the measure is the accuracy over 4 classes.

%Mean distance error (MDE) \tbd research it to check whether it's necessary to use this

\renewcommand{\arraystretch}{1.5}
\begin{table}[ht]
	\centering
	\caption{Metrics for six tasks}
	\label{my-label}
	\begin{tabular}{|m{6cm}|m{6cm}|}
		\hline
		Task                    & Metric         \\ \hline
		3D vehicle detection    & $IoU \geq 0.5$ \\  \hline
		3D localization         &$\bf \lVert \overline t_{pre} - \overline t_{gt} \rVert$ $< 1 / 2$ meters   \\ \hline
		3D orientation estimation  & $OS$     \\ \hline
		3D dimension estimation & $\left | \frac{d-d_{gt}}{d_{gt}} \right | _{d =\{h,w,l\}}<  20\%$           \\ \hline
		2D part localization    & $ \bf \lVert \overline p_{pre} - \overline p_{gt} \rVert $$<  20$ pixels      \\ \hline
		2D part visibility      &     $V_{pre} = V_{gt}$         \\ \hline
	\end{tabular}
\end{table}
