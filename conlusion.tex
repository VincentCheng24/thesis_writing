 This thesis deals with the problem of 3D pose estimation of vehicles based on monocular images by using deep neural networks for the application of autonomous driving. We train a deep neural network to predict 2D coordinates and visibility property for characteristic points, as well as a template proximity vector, for a target vehicle. During the inference phase, we first perform template matching to reason out the target vehicle's 3D dimensions and 3D coordinates of the key points with the help of the template proximity vector; and then 2D-3D matching is performed to recover the location and orientation of the vehicle. Combining the network and inference phases, our approach can simultaneously performs 3D vehicle detection, 3D localization, 3D orientation estimation, 3D dimension estimation, 2D part localization, and parts visibility characterization for a vehicle in a 2D bounding box patch. 
 
Our work achieves state-of-the-art performance on six tasks. It outperforms most monocular methods recorded in KITTI 3D object detection competition on the most important task, 3D vehicle detection. Besides, our network can predict 2D coordinates and template proximity vector for highly occluded or truncated vehicles and therefore, our approach can perform template matching and 2D-3D matching to recover their 3D bounding boxes. Moreover, unlike other methods, our approach can provide more detailed information of the detected vehicle, \eg 2D part location and 2D part visibility, which are useful for autonomous driving applications to gain a more  fine-grained perception. Finally, the runtime for our approach is at real time level, \textit{ca.} 0.02s per image on one GeForce GTX TITAN X GPU. 

In order to maximize the effective capacity of the model, we thoroughly research and evaluate the key design alternatives and hyperparameters, including loss function, loss weight, learning rate, weight decay, base net feature extractor, model selection strategy, and number of points used in a model. Based on these experiments, the performance of our approach is boosted steadily and finally reaches the state-of-the-art level.

Finally, we present results of these six tasks and compare them with other methods. We also describe the deficiencies of our approach, analyse their causes, and propose possible solutions which are the directions of future improvements. 